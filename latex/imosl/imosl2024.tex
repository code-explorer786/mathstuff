\documentclass{article}

\usepackage[a4paper]{geometry}
\usepackage{amsmath}
\usepackage{amssymb}
\usepackage{amsfonts}
\usepackage{varwidth}
\usepackage{hyperref}
\usepackage{tikz}
\usepackage{tikz-cd}
\usepackage{gensymb}
\usepackage{mathtools}

\usepackage{pgfplots}
\pgfplotsset{compat=1.15}
\usepackage{mathrsfs}
\usetikzlibrary{arrows}

\DeclarePairedDelimiter\ceil{\lceil}{\rceil}
\DeclarePairedDelimiter\floor{\lfloor}{\rfloor}

\title{IMOSL 2024 attempt}
\author{ASimpleBeginner}

\begin{document}
% I should learn how to rice my LaTeX soon...
\maketitle

\section{A1}
Notice that, with respect mod \(n\), the entire thing is periodic with period 2. \\
If \(\alpha = \alpha' + 2\), then
\begin{align*}
	\floor{\alpha} + \dots + \floor{n \alpha}
		&= \floor{\alpha' + 2} + \dots + \floor{n(\alpha' + 2)} \\
		&= \floor{\alpha'} + 2 + \dots + \floor{n\alpha'} + 2n \\
		&\equiv \floor{\alpha'} + \dots + \floor{n\alpha} \pmod{n} 
\end{align*}
Thus we can WLOG check for \(0 \le \alpha < 2\) and finish off later.

Picking \(n := 2\) gives us that \(0 \le \alpha < \frac{1}{2} \; \lor \; \frac{3}{2} \le \alpha < 2\)

We prove that \(\alpha\) must equal 0 in this case. (After WLOG this means that \(\alpha\) must be an even integer.)

If \(\alpha = 0\) we get

\begin{align*}
	\floor{\alpha} + \dots + \floor{n \alpha}
		&= 0 + \dots + 0 = 0
\end{align*}

and thus the sum is divisible by \(n\) for all \(n\).

If \(0 < \alpha < \frac{1}{2}\) we can pick
\[n := \ceil*{\frac{1}{a}} \ge 2\]
such that for all integers \(1 \le k < n\), \(\floor{k \alpha} = 0\), and for \(n\), \(\floor{n \alpha} = 1\) such that
\begin{align*}
	n \nmid \floor{\alpha} + \dots + \floor{(n-1) \alpha} + \floor{n \alpha} = 0 + \dots + 0 + 1 = 1 \\
\end{align*}

We prove this fact. Forall \(1 \le k < n\) --- remembering \(0 < \alpha \le \frac{1}{2}\),
\[\floor{k\alpha} = 0 \; \Leftrightarrow \; 0 \le k\alpha < 1\]

The first inequality is trivial; the second can be proved as follows:

\begin{align*}
	&k\alpha \le \left(\ceil*{\frac{1}{a}} - 1\right)\alpha \le 1\\
	\Leftarrow \; &\ceil*{\frac{1}{a}} - 1 \le \frac{1}{a} \\
	\Leftarrow \; &\ceil*{\frac{1}{a}} \le \frac{1}{a} + 1 \\
	\Leftarrow \; &\text{evident}
\end{align*}

The fact that \(\floor{n\alpha} = 1\) is evident.

Okay, now if \(\frac{3}{2} \le \alpha < 2\) then we can pick
\[n := \ceil*{\frac{1}{2-a}} \ge 2\]
such that for all integers \(1 \le k < n\), \(\floor{k\alpha} = k - 1\), but for \(n\), \(\floor{n\alpha} = n - 2\) such that
\[n \nmid \floor{\alpha} + \dots + \floor{n\alpha} = n^2 - 1\]


\end{document}